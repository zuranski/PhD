
First, I would like to express deep gratitude to my principal advisor Dan Marlow. His patience,
wisdom, and guidance combined with good manners have been invaluable. 
I am also indebted to Jim Olsen who introduced me to high energy physics and clarified the workings
of physics analysis.


This dissertation has been completed in a pleasant and humane environment. This is mainly
thanks to my collaborators, the
long-lived group conveners Loic Quertenmont, Daniele del~Re, John Paul Chou and Steve Worm,
as well as the people with whom I could discuss the very details of the analysis strategy,
Dan Marlow, Ian Tomalin, Emyr Clement and Paul Lujan.

I learned more than I can recall from conversations, help, and support of other fellow
physics students and postdocs. I would like to specially thank Edward Laird who introduced
me to python and fixed gear bicycles. I cannot omit my fellow roommate Halil Saka, he has always been a good companion,
willing to discuss the physics issues or relax with the help of his inexhaustible movie collection.
There is too many to name, but I would also like to thank Michael Mooney, Adam Hunt, Edmund Berry, Xiaohang Quan, Jeroen Hegeman, Paul Lujan, Seth Zenz,
Rafal Staszewski, Maciej Trzebi\'nski and Agnieszka Dziurda. 

In a broad perspective I am particularly grateful to my parents, Janka and Marek, for their tremendous effort in my upbringing and education.
 Thanks to them I've been exposed to science and nature for as long as I can remember. Some of their explanations of the physical phenomena that I remember from childhood,
i.e. the reflections from mirrors, the relativity principle, the Doppler effect etc.
are still the best explanations I can find for them.

Last but not least, I would like to thank all the other people with whom I have shared 
this five years for making it an exciting endeavour, in particular Marie-Luise Menzel and Agata
Witek who continuously encouraged me to explore
life outside of physics.
