\section{Trigger}

The high instantaneous luminosity at the LHC together with the storage space available for
CMS data necessitate an existence of a trigger.
 The aim of the trigger is to reduce the rate
of collision events from 40 MHz, as delivered by the LHC, to $\sim500$ Hz that will be stored
for further analysis, while keeping the events of potential physics interest.
The details of the CMS trigger design and implementation are described in \cite{Cittolin:578006},
only the basic concepts are explained here. 

The CMS trigger system consists of two parts: "Level One" (L1) and "High Level" (HLT).
The L1 is a set of electronics operating at 40 MHz and has a latency of 3~$\mu s$. During this
time only simplified 
objects are reconstructed from the detector readouts using fast electronics.
 They include a L1 muon, photon, 
electron, or jet candidates, as well as a sum of all jets in the event and missing transverse energy.
Only the above objects, or combinations of them, are used to make an acceptance decision. 
The maximal allowed output
rate of the L1 trigger part is 100 kHz, at which the events are processed by the HLT.
The HLT processes events with software reconstruction sequences that closely approximate
the offline reconstruction, while the mean processing time per event is 50ms. There is about 
500 various HLT triggers that each can make an acceptance decision based 
on the reconstructed objects
(jets, leptons, missing energy). The number of trigger paths approximately corresponds to the
number of various searches and measurements pursued by the CMS collaboration.
However, in plenty of cases the physics objects,
upon which the decisions are made, overlap between the triggers in which case the reconstruction 
is performed only once.


\subsection{Trigger for long-lived particles decaying to dijets}


