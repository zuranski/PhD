\section{Trigger}

The high instantaneous luminosity at the LHC together with limitations on the storage space available for
CMS data necessitate the existence of a trigger.
 The aim of the trigger is to reduce the rate
of collision events from 40 MHz, as delivered by the LHC, to $\sim500$ Hz that will be stored
for further analysis, while keeping the events of potential physics interest.
The details of the CMS trigger design and implementation are described in \cite{Cittolin:578006},
only the basic concepts are explained here. 

The CMS trigger system consists of two parts: ``Level One" (L1) and ``High Level" (HLT).
The L1 is a set of electronics operating at 40 MHz and has a latency of 3~$\mu s$. During this
time only simplified 
objects are reconstructed from the detector readouts using fast electronics.
 They include L1 muon, photon, 
electron, or jet candidates, as well as an energy sum of all jets in the event 
and missing transverse energy.
Only the objects above, or combinations of them, are used to make acceptance decisions. 
The maximum allowed output
rate of the L1 trigger part is 100 kHz. Events passing the L1 requirements are processed by the
HLT with 
software reconstruction sequences that closely approximate
the offline reconstruction. The mean processing time per event is 50~ms. 
There are about 
500 various HLT trigger paths, each of which can make an acceptance decision based 
on the reconstructed objects
(jets, leptons, missing energy). The number of trigger paths approximately corresponds to the
number of searches and measurements pursued by the CMS collaboration.
In many of cases, the physics objects
upon which the decisions are made overlap between the triggers, in which case the reconstruction 
is only performed once.


\subsection{Trigger for long-lived particles decaying to dijets}
\label{subsec:trigger}

A dedicated trigger has been designed to accept events that contain long-lived particles 
decaying to dijets. 
The underlying idea behind such a trigger is not to reconstruct dijets
associated with a displaced vertex, because such a procedure is too complicated to be executed 
at the HLT stage, but rather to reject events where all the jets are produced promptly at the collision
point which is much simpler. The rate limitations (up to 1~Hz imposed by the CMS collaboration)
 together with the limited execution time (up to 5~ms on average for a single trigger path)
resulted in the trigger that consists of the following filters:

\begin{enumerate}
 \item Scalar transverse energy sum of all 
the L1 jets in the event, $H_T$, above 150\GeV, which is the lowest threshold allowed by L1
rate limitations;
 \item Scalar transverse energy sum of all 
the HLT jets in the event, $H_T$, above 300\GeV. The jets at L1 are determined with large
uncertainties. Imposing a requirement twice as large at the HLT guarantees very high L1 efficiency and 
therefore reduces the dependence of the further analysis on the L1 trigger performance;
\item At least two jets that have transverse momenta, $p_T>60$\GeV and pseudorapidity
$|\eta|<2$. This requirement selects jets that are central and well within the acceptance 
of the CMS tracker;
\item At least two of the jets selected in step 3 are required to have not more than two tracks
that have impact parameters smaller than 300\micron. This requirement rejects many promptly
produced jets, having a majority of their tracks with small impact parameters;
\item For at least two of the jets selected in step 4 the jet energy fraction carried by the tracks
that have transverse impact parameters smaller than 500\micron is required to be smaller than
15\%. This requirement additionally suppresses promptly produced jets. 
\end{enumerate}

In addition to the trigger just described, a control trigger has been designed which
consists of the same steps, but modified such that the steps 3 through 5 require that only one jet 
 fulfills the corresponding requirement. The two triggers are then called {\it double displaced jet} and 
{\it single displaced jet} triggers. Due to HLT rate limitations the {\it single} trigger
was prescaled.
Both triggers were active during the entire 2012 LHC run and the number of events collected by CMS
together with the integrated luminosity are summarized in Table \ref{tab:triggerEvents}.
\begin{table}[hbtp]
\begin{center}
\begin{tabular}{l c c c }
\hline
trigger name & prescale factor & \lumi [\fbinv] & N events [1e6] \\
\hline
single displaced jet & 100-120 & 0.18 & 0.5\\
double displaced jet & 1 & 18.5 & 1.9\\
\hline
\end{tabular}
\end{center}
\caption{Displaced jet triggers active in 2012 LHC run.\label{tab:triggerEvents}}
\end{table}
Data collected by the double displaced jet trigger, where the presence of two triggering jets is required,
 is used to search for our signal,
while data collected by the single jet one is used as a prescaled control sample.

