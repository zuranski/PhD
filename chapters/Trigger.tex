\section{Trigger}

The high instantaneous luminosity at the LHC together with the storage space available for
CMS data necessitate an existence of a trigger.
 The aim of the trigger is to reduce the rate
of collision events from 40 MHz, as delivered by the LHC, to $\sim500$ Hz that will be stored
for further analysis, while keeping the events of potential physics interest.
The details of the CMS trigger design and implementation are described in \cite{Cittolin:578006},
only the basic concepts are explained here. 

The CMS trigger system consists of two parts: "Level One" (L1) and "High Level" (HLT).
The L1 is a set of electronics operating at 40 MHz and has a latency of 3~$\mu s$. During this
time only simplified 
objects are reconstructed from the detector readouts using fast electronics.
 They include a L1 muon, photon, 
electron, or jet candidates, as well as a sum of all jets in the event and missing transverse energy.
Only the above objects, or combinations of them, are used to make an acceptance decision. 
The maximal allowed output
rate of the L1 trigger part is 100 kHz, at which the events are processed by the HLT.
The HLT processes events with software reconstruction sequences that closely approximate
the offline reconstruction, while the mean processing time per event is 50ms. There is about 
500 various HLT triggers that each can make an acceptance decision based 
on the reconstructed objects
(jets, leptons, missing energy). The number of trigger paths approximately corresponds to the
number of various searches and measurements pursued by the CMS collaboration.
However, in plenty of cases the physics objects,
upon which the decisions are made, overlap between the triggers in which case the reconstruction 
is performed only once.


\subsection{Trigger for long-lived particles decaying to dijets}

A dedicated trigger has been designed to accept events that contain long-lived particles 
decaying to dijets. 
The underlying idea behind such a trigger is not to reconstruct dijets
associated with a displaced vertex, because such a procedure is too complicated to be executed 
at HLT, but rather to reject events where all the jets are produced promptly at the collision
point which is much simpler. The rate limitations (up to 1~Hz imposed by the CMS collaboration)
 together with the limited execution time (up to 5~ms on average for a single trigger)
resulted in the trigger that consists of the following filters:

\begin{enumerate}
 \item Scalar transverse energy sum of all 
the L1 jets in the event, $H_T$, above 150\GeV, which is the lowest threshold allowed by CMS;
 \item Scalar transverse energy sum of all 
the HLT jets in the event, $H_T$, above 300\GeV. The jets at L1 are determined with large
uncertainties, imposing a requirement twice as large at HLT guarantees very high L1 efficiency and 
therefore reduces the dependence of the further analysis on the L1 trigger performance and
simulation;
\item At least two jets that have transverse momenta, $p_T>$~60\GeV and pseudorapidity
$|\eta|<$2. This requirement selects jets that are central and well within the acceptance 
of the CMS tracker;
\item At least two of the jets selected in step 3 are required to have not more than 2 tracks
that have impact parameters smaller than 300~$\mu m$. This requirement rejects plenty of promptly
produced jets for which majority of their tracks have smaller impact parameters;
\item For at least two of the jets selected in step 4 the jet energy fraction carried by the tracks
that have transverse impact parameters smaller than 500~$\mu m$ is required to be smaller than
15\%. This requirement additionally rejects promptly produced jets that passed step 4. 
\end{enumerate}

Additionally to the trigger just described, a control trigger has been designed which
consists of the same steps, but the steps 3-5 require only one jet to fulfill
the corresponding requirement. The two triggers are then called a {\it double displaced jet} and 
a {\it single displaced jet} triggers, while due to rate limitations the {\it single} trigger
was prescaled.  
Both triggers were active during the 2012 LHC run and the number of events collected by CMS
together with the integrated luminosity are summarized in Table \ref{tab:triggerEvents}.
\begin{table}[hbtp]
\begin{center}
\begin{tabular}{l c c c }
\hline
trigger name & prescale factor & \lumi [\fbinv] & N events [1e6] \\
\hline
single displaced jet & 100-120 & 0.18 & 0.5\\
double displaced jet & 1 & 18.5 & 1.9\\
\hline
\end{tabular}
\end{center}
\caption{Displaced jet triggers active in 2012 LHC run.\label{tab:triggerEvents}}
\end{table}
Data collected by the double displaced jet trigger, where presence of two triggering jets is required,
 is used to search for our signal,
while data collected by the single jet one, where presence of only one triggering jet is required, is used as a control sample.

