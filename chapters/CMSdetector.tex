\section{The CMS detector}

The central feature of the Compact Muon Solenoid (CMS) apparatus is a superconducting solenoid of 6\unit{m} internal diameter. Within the superconducting solenoid volume are a silicon pixel and strip tracker, a lead tungstate crystal electromagnetic calorimeter (ECAL), and a brass/scintillator hadron calorimeter (HCAL). Muons are measured in gas-ionization detectors embedded in the steel return yoke outside the solenoid. Extensive forward calorimetry complements the coverage provided by the barrel and endcap detectors. 

The inner tracker measures charged particles within the pseudorapidity range $\abs{\eta}< 2.5$. It consists of 1440 silicon pixel and 15\,148 silicon strip detector modules and is located in the 3.8\unit{T} field of the superconducting solenoid. It provides an impact parameter resolution of ${\sim}15\mum$ and a transverse momentum (\pt) resolution of about 1.5\% for 100\GeV particles. 
The track reconstruction algorithms are able to reconstruct displaced tracks with transverse impact
parameters up to ${\approx}30$\,cm from particles decaying up to ${\approx}60$\,cm from the beam line.  The
performance of the track reconstruction algorithms has been studied with data
\cite{Khachatryan:2010pw}. 
%The silicon
%tracker is also used to reconstruct the primary vertices positions with a
%precision of ${\sim}20$~\mum in each dimension.


The global event reconstruction (also called particle-flow event reconstruction~\cite{CMS-PAS-PFT-09-001,CMS-PAS-PFT-10-001}) is designed to reconstruct and identify each particle in the event using an optimized combination of all subdetector information. 
%In this process, the identification of the particle type (photon, electron, muon, charged hadron, neutral hadron) plays an important role in the determination of the particle direction and energy. Photons (\eg coming from \Pgpz\ decays or from electron bremsstrahlung) are identified as ECAL energy clusters not linked to the extrapolation of any charged particle trajectory to the ECAL. Electrons (\eg coming from photon conversions in the tracker material or from \cPqb-hadron semileptonic decays) are identified as a primary charged particle track and potentially many ECAL energy clusters corresponding to this track extrapolation to the ECAL and to possible bremsstrahlung photons emitted along the way through the tracker material. Muons (\eg from \cPqb-hadron semileptonic decays) are identified as a track in the central tracker consistent with either a track or several hits in the muon system, associated with an energy deficit in the calorimeters. Charged hadrons are identified as charged particle tracks neither identified as electrons, nor as muons. Finally, neutral hadrons are identified as HCAL energy clusters not linked to any charged hadron trajectory, or as ECAL and HCAL energy excesses with respect to the expected charged hadron energy deposit. 
For each event, hadronic jets are clustered from these reconstructed particles with the infrared and collinear safe
 anti-$k_\mathrm{t}$ algorithm 
operated with a size parameter $R$ of 0.5. The size parameter requires that all the jet 
particles have $\Delta R \leq 0.5$ relative to the jet momentum vector, where
 $\Delta R=\sqrt{(\Delta\phi)^2 + (\Delta\eta)^2}$ \cite{Cacciari:2008gp}. 
The jet momentum is determined as the vectorial sum of all particle momenta in this jet.
For jets originating at the event primary vertex the jet momentum is found in the simulation to be within 5\% to 10\% of the true momentum over 
the whole \pt spectrum and detector acceptance. 
When the jet origin is significantly displaced from the event primary vertex, the reduced 
charged particle efficiency results in additional underestimation of the jet momentum. 
For displaced jets originating within the CMS tracker the jet momentum is underestimated in the simulation by up to 10\%.
%Jet energy corrections are derived from the simulation, and are confirmed with in situ measurements with the energy balance of dijet and photon+jet events~\cite{CMS-PAS-JME-10-010}. The jet energy resolution amounts typically to 15\% at 10\GeV, 8\% at 100\GeV, and 4\% at 1\TeV. 
