Using data from the CMS detector at the LHC,
a search for long-lived particles, \X, produced in pp collisions at $\sqrt{s} = 8\TeV$ and decaying
to quark-antiquark pairs has been performed. For the first time a purely hadronic signature
of long-lived particles has been explored at CMS.
The observed results are consistent with SM expectations and are used to derive upper
limits on the product of cross section times branching fraction for a scalar particle,
 \Higgs, in the mass range 200 to 1000\GeV, decaying into a pair of \X bosons in the mass
range 50 to 350\GeV, each of which decays to quark-antiquark pairs. For \Higgs masses of 400--1000\GeV, \X
masses of 50--350\GeV, and \X lifetimes of $0.1<c\tau<200$\:cm, the upper limits are typically 0.5--100\:fb.
 For a $\Higgs$ mass of 200\GeV, the corresponding limits are in the range of 0.09 to 0.2\:pb for \X
lifetimes of 0.2 to 10\:cm. These are the most stringent limits in this channel to date.
In addition, the search allows for interpretation in terms of other models that predict 
the existence of massive long-lived particles with at least two hadronic
jets among their decay products.
