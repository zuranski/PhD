%\section{Introduction}
\label{chap:Introduction}

Several models of new physics predict the existence of massive, long-lived particles which could
manifest themselves through their non-prompt decays to jets. Such scenarios arise, for example,
in various supersymmetric (SUSY) scenarios such as ``split SUSY''
\cite{Hewett:2004nw} or SUSY with very weak R-parity violation \cite{Barbier:2004ez}, ``hidden valley'' models \cite{Han:2007ae}, and \Zprime models
that contain long-lived neutrinos~\cite{Basso:2008iv}.

This Letter presents the first search using data from the Compact Muon
Solenoid (CMS) for massive, long-lived exotic particles \X that decay 
to quark-anitquark pairs (\qq). Quarks will manifest themselves as jets in the CMS detector. 
We therefore search for events
containing a pair of jets originating from a common secondary
vertex within the volume of the CMS tracker,
that is significantly transversely displaced from the event primary vertex.
This topological signature has the potential to provide clear evidence for
physics beyond the standard model (SM). It is also very powerful in suppressing backgrounds from 
standard model processes.

While the analysis presented here is a search for any heavy particle that decays into a pair of jets
 at a displaced vertex, it is useful to have a benchmark model.
We use a specific model of a long-lived, spinless, neutral
exotic particle \X which decays to $\qq$. In this 
model, the \X is pair-produced in the decay of a non-SM Higgs boson, i.e.  \Higgs~$\rightarrow
2$\X~, \X~$\rightarrow\qq$ \cite{Strassler:2006ri}, where the Higgs boson is produced through gluon-gluon
fusion. This model predicts up to two displaced
dijet vertices in the tracking volume per event. 

The CDF and D0 collaborations have performed searches for metastable particles decaying to B hadron 
jets \cite{Aaltonen:2011rja, Abazov:2009ik}.
These searches are sensitive to a smaller kinematic phase space region than CMS, but explore
lower masses of the exotic particles. The ATLAS collaboration
has performed searches that are sensitive to decay lengths between 1-20\unit{m} by exploiting the ATLAS muon
 spectrometer \cite{ATLAS:2012av}, whilst the search presented here is sensitive to decay lengths below 1 metre.
 The ATLAS search required the long-lived particles to be pair-produced,
while our search  
is also sensitive to single or associated production. The CMS collaboration performed
a search for long-lived particles in a phase space region analogous to the region explored in the presented search,
 however using the leptonic decay channels \cite{Chatrchyan:2012jna}.
  


